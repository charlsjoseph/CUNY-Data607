\documentclass[]{article}
\usepackage{lmodern}
\usepackage{amssymb,amsmath}
\usepackage{ifxetex,ifluatex}
\usepackage{fixltx2e} % provides \textsubscript
\ifnum 0\ifxetex 1\fi\ifluatex 1\fi=0 % if pdftex
  \usepackage[T1]{fontenc}
  \usepackage[utf8]{inputenc}
\else % if luatex or xelatex
  \ifxetex
    \usepackage{mathspec}
  \else
    \usepackage{fontspec}
  \fi
  \defaultfontfeatures{Ligatures=TeX,Scale=MatchLowercase}
\fi
% use upquote if available, for straight quotes in verbatim environments
\IfFileExists{upquote.sty}{\usepackage{upquote}}{}
% use microtype if available
\IfFileExists{microtype.sty}{%
\usepackage{microtype}
\UseMicrotypeSet[protrusion]{basicmath} % disable protrusion for tt fonts
}{}
\usepackage[margin=1in]{geometry}
\usepackage{hyperref}
\hypersetup{unicode=true,
            pdftitle={Data607\_week7\_assignment},
            pdfauthor={Charls Joseph},
            pdfborder={0 0 0},
            breaklinks=true}
\urlstyle{same}  % don't use monospace font for urls
\usepackage{color}
\usepackage{fancyvrb}
\newcommand{\VerbBar}{|}
\newcommand{\VERB}{\Verb[commandchars=\\\{\}]}
\DefineVerbatimEnvironment{Highlighting}{Verbatim}{commandchars=\\\{\}}
% Add ',fontsize=\small' for more characters per line
\usepackage{framed}
\definecolor{shadecolor}{RGB}{248,248,248}
\newenvironment{Shaded}{\begin{snugshade}}{\end{snugshade}}
\newcommand{\KeywordTok}[1]{\textcolor[rgb]{0.13,0.29,0.53}{\textbf{#1}}}
\newcommand{\DataTypeTok}[1]{\textcolor[rgb]{0.13,0.29,0.53}{#1}}
\newcommand{\DecValTok}[1]{\textcolor[rgb]{0.00,0.00,0.81}{#1}}
\newcommand{\BaseNTok}[1]{\textcolor[rgb]{0.00,0.00,0.81}{#1}}
\newcommand{\FloatTok}[1]{\textcolor[rgb]{0.00,0.00,0.81}{#1}}
\newcommand{\ConstantTok}[1]{\textcolor[rgb]{0.00,0.00,0.00}{#1}}
\newcommand{\CharTok}[1]{\textcolor[rgb]{0.31,0.60,0.02}{#1}}
\newcommand{\SpecialCharTok}[1]{\textcolor[rgb]{0.00,0.00,0.00}{#1}}
\newcommand{\StringTok}[1]{\textcolor[rgb]{0.31,0.60,0.02}{#1}}
\newcommand{\VerbatimStringTok}[1]{\textcolor[rgb]{0.31,0.60,0.02}{#1}}
\newcommand{\SpecialStringTok}[1]{\textcolor[rgb]{0.31,0.60,0.02}{#1}}
\newcommand{\ImportTok}[1]{#1}
\newcommand{\CommentTok}[1]{\textcolor[rgb]{0.56,0.35,0.01}{\textit{#1}}}
\newcommand{\DocumentationTok}[1]{\textcolor[rgb]{0.56,0.35,0.01}{\textbf{\textit{#1}}}}
\newcommand{\AnnotationTok}[1]{\textcolor[rgb]{0.56,0.35,0.01}{\textbf{\textit{#1}}}}
\newcommand{\CommentVarTok}[1]{\textcolor[rgb]{0.56,0.35,0.01}{\textbf{\textit{#1}}}}
\newcommand{\OtherTok}[1]{\textcolor[rgb]{0.56,0.35,0.01}{#1}}
\newcommand{\FunctionTok}[1]{\textcolor[rgb]{0.00,0.00,0.00}{#1}}
\newcommand{\VariableTok}[1]{\textcolor[rgb]{0.00,0.00,0.00}{#1}}
\newcommand{\ControlFlowTok}[1]{\textcolor[rgb]{0.13,0.29,0.53}{\textbf{#1}}}
\newcommand{\OperatorTok}[1]{\textcolor[rgb]{0.81,0.36,0.00}{\textbf{#1}}}
\newcommand{\BuiltInTok}[1]{#1}
\newcommand{\ExtensionTok}[1]{#1}
\newcommand{\PreprocessorTok}[1]{\textcolor[rgb]{0.56,0.35,0.01}{\textit{#1}}}
\newcommand{\AttributeTok}[1]{\textcolor[rgb]{0.77,0.63,0.00}{#1}}
\newcommand{\RegionMarkerTok}[1]{#1}
\newcommand{\InformationTok}[1]{\textcolor[rgb]{0.56,0.35,0.01}{\textbf{\textit{#1}}}}
\newcommand{\WarningTok}[1]{\textcolor[rgb]{0.56,0.35,0.01}{\textbf{\textit{#1}}}}
\newcommand{\AlertTok}[1]{\textcolor[rgb]{0.94,0.16,0.16}{#1}}
\newcommand{\ErrorTok}[1]{\textcolor[rgb]{0.64,0.00,0.00}{\textbf{#1}}}
\newcommand{\NormalTok}[1]{#1}
\usepackage{graphicx,grffile}
\makeatletter
\def\maxwidth{\ifdim\Gin@nat@width>\linewidth\linewidth\else\Gin@nat@width\fi}
\def\maxheight{\ifdim\Gin@nat@height>\textheight\textheight\else\Gin@nat@height\fi}
\makeatother
% Scale images if necessary, so that they will not overflow the page
% margins by default, and it is still possible to overwrite the defaults
% using explicit options in \includegraphics[width, height, ...]{}
\setkeys{Gin}{width=\maxwidth,height=\maxheight,keepaspectratio}
\IfFileExists{parskip.sty}{%
\usepackage{parskip}
}{% else
\setlength{\parindent}{0pt}
\setlength{\parskip}{6pt plus 2pt minus 1pt}
}
\setlength{\emergencystretch}{3em}  % prevent overfull lines
\providecommand{\tightlist}{%
  \setlength{\itemsep}{0pt}\setlength{\parskip}{0pt}}
\setcounter{secnumdepth}{0}
% Redefines (sub)paragraphs to behave more like sections
\ifx\paragraph\undefined\else
\let\oldparagraph\paragraph
\renewcommand{\paragraph}[1]{\oldparagraph{#1}\mbox{}}
\fi
\ifx\subparagraph\undefined\else
\let\oldsubparagraph\subparagraph
\renewcommand{\subparagraph}[1]{\oldsubparagraph{#1}\mbox{}}
\fi

%%% Use protect on footnotes to avoid problems with footnotes in titles
\let\rmarkdownfootnote\footnote%
\def\footnote{\protect\rmarkdownfootnote}

%%% Change title format to be more compact
\usepackage{titling}

% Create subtitle command for use in maketitle
\newcommand{\subtitle}[1]{
  \posttitle{
    \begin{center}\large#1\end{center}
    }
}

\setlength{\droptitle}{-2em}

  \title{Data607\_week7\_assignment}
    \pretitle{\vspace{\droptitle}\centering\huge}
  \posttitle{\par}
    \author{Charls Joseph}
    \preauthor{\centering\large\emph}
  \postauthor{\par}
      \predate{\centering\large\emph}
  \postdate{\par}
    \date{October 11, 2018}

\usepackage{booktabs}
\usepackage{longtable}
\usepackage{array}
\usepackage{multirow}
\usepackage[table]{xcolor}
\usepackage{wrapfig}
\usepackage{float}
\usepackage{colortbl}
\usepackage{pdflscape}
\usepackage{tabu}
\usepackage{threeparttable}
\usepackage{threeparttablex}
\usepackage[normalem]{ulem}
\usepackage{makecell}

\begin{document}
\maketitle

\subsection{Assignment -- Working with XML and JSON in
R}\label{assignment-working-with-xml-and-json-in-r}

Here is the assignmnet work statment:

Pick three of your favorite books on one of your favorite subjects. At
least one of the books should have more than one author. For each book,
include the title, authors, and two or three other attributes that you
find interesting. Take the information that you've selected about these
three books, and separately create three files which store the book's
information in HTML (using an html table), XML, and JSON formats (e.g.
``books.html'', ``books.xml'', and ``books.json''). To help you better
understand the different file structures, I'd prefer that you create
each of these files ``by hand'' unless you're already very comfortable
with the file formats. Write R code, using your packages of choice, to
load the information from each of the three sources into separate R data
frames. Are the three data frames identical? Your deliverable is the
three source files and the R code. If you can, package your assignment
solution up into an .Rmd file and publish to rpubs.com. {[}This will
also require finding a way to make your three text files accessible from
the web{]}.

\subsubsection{Libraries Requried}\label{libraries-requried}

\begin{Shaded}
\begin{Highlighting}[]
\KeywordTok{library}\NormalTok{(XML)    }
\KeywordTok{library}\NormalTok{(RCurl)  ## required for GetURL method }
\end{Highlighting}
\end{Shaded}

\begin{verbatim}
## Loading required package: bitops
\end{verbatim}

\begin{Shaded}
\begin{Highlighting}[]
\KeywordTok{library}\NormalTok{(jsonlite)}
\KeywordTok{library}\NormalTok{(kableExtra)}
\KeywordTok{library}\NormalTok{(knitr)}
\end{Highlighting}
\end{Shaded}

\subsubsection{Reading html file and
parsing.}\label{reading-html-file-and-parsing.}

htmlParse is used to parse the html content and invoked readHTMLTable to
extract the html table.

\begin{Shaded}
\begin{Highlighting}[]
\NormalTok{html.url <-}\StringTok{ "https://raw.githubusercontent.com/charlsjoseph/CUNY-Data607/master/week7/books.html"}

\NormalTok{html.url.content=}\StringTok{ }\KeywordTok{getURL}\NormalTok{(html.url)}
\NormalTok{books.html <-}\StringTok{ }\KeywordTok{htmlParse}\NormalTok{(}\DataTypeTok{file=}\NormalTok{ html.url.content )}


\NormalTok{books.html.df <-}\StringTok{ }\KeywordTok{as.data.frame}\NormalTok{(}\KeywordTok{readHTMLTable}\NormalTok{(books.html))}
\NormalTok{books.html.df}
\end{Highlighting}
\end{Shaded}

\begin{verbatim}
##                 NULL.Book.Name                NULL.Author
## 1                 The Talisman Stephen King, Peter Straub
## 2                    The Pearl             John Steinbeck
## 3 POEM OF THE MAN-GOD Volume 1             Maria Valtorta
##                  NULL.Publisher NULL.Year.Published NULL.Pages
## 1                  Pocket Books                1984        944
## 2                 Penguin Books                1993         96
## 3 Centro Editoriale Valtortiano                2000        995
##   NULL.Language
## 1       English
## 2       English
## 3       English
\end{verbatim}

\begin{Shaded}
\begin{Highlighting}[]
\KeywordTok{head}\NormalTok{(books.html.df)}
\end{Highlighting}
\end{Shaded}

\begin{verbatim}
##                 NULL.Book.Name                NULL.Author
## 1                 The Talisman Stephen King, Peter Straub
## 2                    The Pearl             John Steinbeck
## 3 POEM OF THE MAN-GOD Volume 1             Maria Valtorta
##                  NULL.Publisher NULL.Year.Published NULL.Pages
## 1                  Pocket Books                1984        944
## 2                 Penguin Books                1993         96
## 3 Centro Editoriale Valtortiano                2000        995
##   NULL.Language
## 1       English
## 2       English
## 3       English
\end{verbatim}

\begin{Shaded}
\begin{Highlighting}[]
\CommentTok{# print datafrane using kabble}

\NormalTok{books.html.df }\OperatorTok
\StringTok{  }\KeywordTok{kable}\NormalTok{() }\OperatorTok
\StringTok{  }\KeywordTok{kable_styling}\NormalTok{()}
\end{Highlighting}
\end{Shaded}

\begin{table}[H]
\centering
\begin{tabular}{l|l|l|l|l|l}
\hline
NULL.Book.Name & NULL.Author & NULL.Publisher & NULL.Year.Published & NULL.Pages & NULL.Language\\
\hline
The Talisman & Stephen King, Peter Straub & Pocket Books & 1984 & 944 & English\\
\hline
The Pearl & John Steinbeck & Penguin Books & 1993 & 96 & English\\
\hline
POEM OF THE MAN-GOD Volume 1 & Maria Valtorta & Centro Editoriale Valtortiano & 2000 & 995 & English\\
\hline
\end{tabular}
\end{table}

\subsubsection{Reading XML file and
parsing.}\label{reading-xml-file-and-parsing.}

xmlParse is used to parse the xml content and invoked xmlToDataFrame to
convert the content into dataframe

\begin{Shaded}
\begin{Highlighting}[]
\NormalTok{xml.url <-}\StringTok{ "https://raw.githubusercontent.com/charlsjoseph/CUNY-Data607/master/week7/books.xml"}

\NormalTok{xml.url.content=}\StringTok{ }\KeywordTok{getURL}\NormalTok{(xml.url)}


\NormalTok{xml.url.content <-}\StringTok{ }\KeywordTok{xmlParse}\NormalTok{(}\DataTypeTok{file=}\NormalTok{xml.url.content)}
\NormalTok{xml.url.df <-}\StringTok{ }\KeywordTok{xmlToDataFrame}\NormalTok{(xml.url.content)}
\NormalTok{xml.url.df}
\end{Highlighting}
\end{Shaded}

\begin{verbatim}
##                      Book_Name                     Author
## 1                 The Talisman Stephen King, Peter Straub
## 2                        Pearl             John Steinbeck
## 3 POEM OF THE MAN-GOD Volume 1             Maria Valtorta
##                       Publisher Year_Published Pages Language
## 1                  Pocket Books           1984   944  English
## 2                 Penguin Books           1993    96  English
## 3 Centro Editoriale Valtortiano           1990   995  English
\end{verbatim}

\begin{Shaded}
\begin{Highlighting}[]
\CommentTok{# print datafrane using kabble}
\NormalTok{xml.url.df }\OperatorTok
\StringTok{  }\KeywordTok{kable}\NormalTok{() }\OperatorTok
\StringTok{  }\KeywordTok{kable_styling}\NormalTok{()}
\end{Highlighting}
\end{Shaded}

\begin{table}[H]
\centering
\begin{tabular}{l|l|l|l|l|l}
\hline
Book\_Name & Author & Publisher & Year\_Published & Pages & Language\\
\hline
The Talisman & Stephen King, Peter Straub & Pocket Books & 1984 & 944 & English\\
\hline
Pearl & John Steinbeck & Penguin Books & 1993 & 96 & English\\
\hline
POEM OF THE MAN-GOD Volume 1 & Maria Valtorta & Centro Editoriale Valtortiano & 1990 & 995 & English\\
\hline
\end{tabular}
\end{table}

\subsubsection{Reading JSON file and
parsing.}\label{reading-json-file-and-parsing.}

fromJSON is used to parse and convert into the json df

\begin{Shaded}
\begin{Highlighting}[]
\NormalTok{json.url <-}\StringTok{ "https://raw.githubusercontent.com/charlsjoseph/CUNY-Data607/master/week7/books.json"}
\NormalTok{json.url.content=}\StringTok{ }\KeywordTok{getURL}\NormalTok{(json.url)}
\NormalTok{json.url.df <-}\StringTok{ }\KeywordTok{fromJSON}\NormalTok{(json.url.content)}
\NormalTok{json.url.df}
\end{Highlighting}
\end{Shaded}

\begin{verbatim}
## $books
##                       Book_Name                     Author
## 1                  The Talisman Stephen King, Peter Straub
## 2                         Pearl             John Steinbeck
## 3 POEM OF THE MAN-GOD Volume 1              Maria Valtorta
##                       Publisher Year Pages Language
## 1                  Pocket Books 1984   944  English
## 2                 Penguin Books 1993    96  English
## 3 Centro Editoriale Valtortiano 1990   995  English
\end{verbatim}

\begin{Shaded}
\begin{Highlighting}[]
\CommentTok{# print datafrane using kabble}

\NormalTok{json.url.df }\OperatorTok
\StringTok{  }\KeywordTok{kable}\NormalTok{() }\OperatorTok
\StringTok{  }\KeywordTok{kable_styling}\NormalTok{()}
\end{Highlighting}
\end{Shaded}

\begin{table}[H]
\centering\begin{table}

\centering
\begin{tabular}[t]{l|l|l|r|r|l}
\hline
Book\_Name & Author & Publisher & Year & Pages & Language\\
\hline
The Talisman & Stephen King, Peter Straub & Pocket Books & 1984 & 944 & English\\
\hline
Pearl & John Steinbeck & Penguin Books & 1993 & 96 & English\\
\hline
POEM OF THE MAN-GOD Volume 1 & Maria Valtorta & Centro Editoriale Valtortiano & 1990 & 995 & English\\
\hline
\end{tabular}
\end{table}
\end{table}

\subsubsection{Conclusion}\label{conclusion}

All three dataframes are almost same other than very minor difference in
the column name. For e.g the column names in html df has a dot, but for
xml and json has \_. Also the column name is prefixed with a NULL.


\end{document}
